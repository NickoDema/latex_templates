\documentclass{ncv}

\usepackage{url}
\usepackage{mwe}
\usepackage{graphbox}

\usepackage{fontawesome}
\usepackage{setspace}
\usepackage{enumitem}
\usepackage{hyperref}

\usepackage[dvipsnames]{xcolor}

\graphicspath{{img/}}
\DeclareGraphicsExtensions{.pdf, .jpg, .png}

\name{Nikolay Dema}
\photo{25_h_2}
\mail{ndema2301@gmail.com}
\github{\href{https://github.com/NickoDema}{github.com/NickoDema}}
\phone{+79816810652}
\youtube{\href{https://www.youtube.com/channel/UCiGwbcr2f2ON77etnY-aVxw}{goo.gl/j6Ac4W}}
\telegram{\href{https://t.me/nicko_dema}{@nicko\_dema}}
\instagram{\href{https://www.instagram.com/nicko_dema/}{@nicko\_dema}}


\begin{document}

\begin{center}
    {\Huge Curriculum Vitae}
\end{center}

\vspace{1cm}

\header

\ 

\section{Education}
\begin{two_col_entry_list}
	\entry
		{2013 -- 2017}
		{\textbf{Bachelor's Degree} in Mechatronics and Robotics, ITMO University, \\ Department of Control Systems and Informatics, St. Petersburg, Russia}
\end{two_col_entry_list}

\ 

\section{Work Experience}
\begin{two_col_entry_list}
    \entryh
		{2021 may -- now}
		{JSC NIIAS, System Architect}
		{}
		{Software architecture improvements and low-level protocols development for on-board control and computing systems of autonomous shunting locomotives.
		}
    \entryh
		{2020 sep. -- now}
		{ScPA StarLine, Tech Lead}
		{}
		{Management of a small R\&D group of 6 employees of \textcolor{MidnightBlue}{\href{https://smartcar.starline.ru/en/oscar-en}{OSCAR}} project.\\[1.5mm]
		\textcolor{MidnightBlue}{\href{https://alpha.starline.ru/}{Alpha}} drive-by-ware kit Launch.\\[1.5mm]
		Organization of joint events and educational programs with local universities.
		}
    \entryh
		{2019 aug. -- 2020 aug.}
		{ScPA StarLine, Robotics Research Engineer}
		{}
		{API design and low-level driver development for brand-new drive-by-wire system for autonomous vehicles. Architecture improvements of self-driving software stack.
		}
    \entryh
		{2018 aug. -- 2019 jul.}
		{goTRG, Robotics Software Engineer}
		{}
		{Development of navigation software from low level hardware drivers and sensor fusion to motion planing and localization for a differential drive AGV.\\[1.5mm]
		Participation in the development of an orchestration level software to control heterogeneous fleet of robots in a logistics warehouses and distribution centers.
% 		as well as human-following functional and a network API for a differential drive AGV.
		}
    \entryh
		{2017 sep. -- 2018 dec.}
		{ITMO University, Research Engineer}
		{}
		{Research conduction on motion planning and control for kinematically redundant manipulators (supported by RSF, project No. 17-79-20341).\\[1.5mm] 
		Development of data encryption protocol for cyber-physical systems based on SCW quantum key distribution (supported by the Government of the Russian Federation, Grant 08-08) 
		}
	\entryh
		{2016 sep. -- 2017 aug.}
		{ITMO University, Laboratory Assistant}
		{}
		{Development of shared control system for a motorized wheelchair to  prevent collisions with obstacles during movement in a complex dynamic environment.\\[1.5mm]
		Participation in the development of the telepresence robot with AR interface, designed a navigation system for remotely operated and autonomous modes.}
\end{two_col_entry_list}

\ 

% \newpage
\section{Technical Skills}
\begin{two_col_entry_list}
	\entry
		{Languages}
		{C++, Python, Matlab, Bash, Latex}
	\entry
		{Tools}
		{Linux, ROS, ROS2, OpenCV, Apollo(CyberRT), Docker, {\O}MQ, Git, CAN, CANOpen, Serial,  Atlassian Stack, Gazebo, V-rep, Stage, SolidWorks, Blender, UE4.}
	\entry
		{Practical\\ experience}
		{
		%\textbf{Hardware:} 
		Kuka YouBot, UR, Festo Robotino, Nanotec, Roboteq, Velodyne, Ouster, Hokuyo, Dynamixel, RealSense, Kobuki, Orbbec.}
	\entry
		{Research\\ Interests}
		{Cognitive neuro-robotics, human-robot interaction, edge computing, computer graphics.}
\end{two_col_entry_list}


\section{Publications}
	\begin{pub_itemize}
        \item \underline{N. U. Dema} and S. A. Kolyubin, \textit{An algorithm of shared control of intellectual wheelchair movement.} Journal of Instrument Engineering, Vol. 61, no. 2, 2018
            
        \item Oleg I. Borisov, Vladislav S. Gromov, Sergey A. Kolyubin, Anton A. Pyrkin, \underline{Nikolay Y. Dema}, Vladimir I. Salikhov, Igor V. Petranevsky, Alexey O. Klyunin,  Alexey A. Bobtsov, \textit{Case study on human-free water heaters production for industry 4.0}, IEEE Industrial Cyber-Physical Systems (ICPS), 2018, pp. 369-374.
            
        \item A. I. Shchekoldin, A. D. Shevyakov, \underline{N. U. Dema} and S. A. Kolyubin. \textit{Adaptive head movements tracking algorithms for AR interface controlled telepresence robot.} IEEE 22nd International Conference on Methods and Models in Automation and Robotics, Miedzyzdroje, 2017, pp. 728-733.
    \end{pub_itemize}
         
         
\section{Communication and Leadership}
\begin{two_col_entry_list}
    \entryhc
        {2021, 2022}
		{Technical Committee member of the \textcolor{MidnightBlue}{\href{https://yandex.ru/profi/}{"Ya - Professional"}}}
		{}
		{Technical preparations of tasks for the Ya - Professional competitors in the robotics subgroup.}
    \entryhc
		{2020}
		{\textcolor{MidnightBlue}{\href{https://robofinist.ru/event/info/media/id/339}{Self-Driving StarLine Competition organization}}}
		{}
		{Technical and organizational preparation of one of the largest student competitions in the field of autonomous transport in Russia.}
    \entryhc
		{2017, 2018}
		{RoboCup@Work Captain of the \textcolor{MidnightBlue}{\href{https://red-itmo.github.io/}{RED Team}}}
		{}
		{Teamwork organization. Development of an orchestration level software and navigation subsystem for an omnidirectional mobile robot.}
    \entryhc
		{2016 sep. -- 2018 dec.}
		{Student robotics laboratory supervisor}
		{}
		{Led many student projects in such topics as mobile robot motion planning and control, computer vision and manipulation.}
\end{two_col_entry_list}


\section{Courses and Qualifications}
\begin{two_col_entry_list}
    \entryc
		{2019 aug.}
		{20th Max Planck Advanced Course on the Foundations of Computer Science (ADFOCS-2019), Saarbrücken, Germany.}
    \entryc
		{2019 jun.}
		{Summer School on Nonlinear and Adaptive Control (SNAC-2019), Saint Petersburg, Russia.  }
    \entryc
		{2018 apr.}
		{The European Embedded Control Institute, International Graduate School on Control (EECI-IGSC-2018-M19), Saint Petersburg, Russia. }
\end{two_col_entry_list}

\section{Honors and Awards}
\begin{two_col_entry_list}
%    \entryc
%		{2019}
%		{Member of the team that won \textit{first place in MOBI Grand Challenge} (Chorus Mobility and Decentralized Technology MGC team). Configured and developed software for "duckietown" platform.}
    \entryc
		{2018}
		{Russian Federation Government Fellowship}
	\entryc
		{2018}
		{Captain of the team that won \textit{second place in "Techno Challenge Smart Welding"}. Developed the concept of a positioning and accuracy control system for locomotive welded parts}
    \entryc
		{2017, 2018}
		{Winner of the Saint Petersburg Government grant competition for
young researchers.}

\end{two_col_entry_list}


\section{Languages}

\begin{tabular}{ l l }
\textbf{Russian} - & \ native\\ 
\textbf{English} - & \ advanced    
\end{tabular}
%\textbf{Russian} - native\\
%\textbf{English} - advanced
%\begin{minipage}[c]{0.3\textwidth}
%	\vspace{-\baselineskip}
%	
%	\section{Research interest}
%	
%	Computational geometry, Navigation, Redundant kinematics,
%\end{minipage}
%\hfill
%\begin{minipage}[c]{0.3\textwidth}
%	\vspace{-\baselineskip} 
%	
%	\section{Hobbies}
%	
%	I love...
%\end{minipage}
%\hspace{1cm}
%\begin{minipage}[c]{0.3\textwidth}
%	\vspace{-\baselineskip} % Required for vertically aligning minipages
%
%	\section{Languages}
%	
%	\textbf{Russian} - native\\
%	\textbf{English} - advanced\\
%	\textbf{Japanese} - rudimentary
%\end{minipage}
%\hfill
\end{document}
