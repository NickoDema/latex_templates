\documentclass{nreport}

\usepackage{cmap}
\usepackage[T2A, T1]{fontenc}
\usepackage[utf8]{inputenc}
\usepackage[russian, english]{babel}

%% temporary 
%\usepackage{ucs}
%% --------------

%% Mathematics
\usepackage{amsmath}
\usepackage{amsthm}
\usepackage{lipsum}
\usepackage{amsfonts}
%% ------------------

%% graphics
\usepackage{wrapfig}
\usepackage{graphicx}
\usepackage{subcaption}
\graphicspath{{pictures/}}
\DeclareGraphicsExtensions{.pdf, .jpg, .png}
%% -----------------------------------------

%% Fonts
%\usepackage{imfellEnglish}

\usepackage{newpxtext,newpxmath}
%\usepackage{newpxmath}
%\usepackage{paratype}
%\usepackage{mathpazo}

%\usepackage{tgpagella}
%% --------------------

%% -
\usepackage{url}      %% Hyperlinks
\usepackage{listings} %% Source code highlighting
\setlength{\emergencystretch}{0.6cm}


\title{Title}
\author{Dema Nikolay}
\date{\today}

\begin{document}

\maketitle
\tableofcontents

\chapter{Introduction}

See \nameref{ch2}

\lipsum[1]

\section{Introduction}

This document is a sample document to test font families and font typefaces. This text uses a different font typeface. This document is a sample document to test font families and font typefaces. This text uses a different font typeface. k

\lipsum[2]

\section{Math}

Let us suppose that $x^2+y^2=z^2$. Then
\begin{equation}
\biggl\langle u\biggm|\sum_{i=1}^nF(e_i,v)e_i\biggr\rangle
    =F\biggl(\sum_{i=1}^n\langle e_i|u\rangle e_i,v\biggr).
\end{equation}

\chapter{Second chapter}\label{ch2}

\lipsum[3]

\section{First section}

\lipsum[4]

$$
\omega = \sum_\alpha^\phi {CRXVB}
$$

\begin{align*}
RQSZ \\
\mathcal{RQSZ} \\
\mathfrak{RQSZ} \\
\mathbb{RQSZ}
\end{align*}

\subsection{First subsection}

\lipsum[5]

\begin{align*}
3x^2 \in R \subset Q \\
\mathnormal{3x^2 \in R \subset Q} \\
\mathrm{3x^2 \in R \subset Q} \\
\mathit{3x^2 \in R \subset Q} \\
\mathbf{3x^2 \in R \subset Q} \\
\mathsf{3x^2 \in R \subset Q} \\
\mathtt{3x^2 \in R \subset Q} 
\end{align*}

\rus{В 1930 году Паули высказал предположение о том, что может существовать легкая электрически нейтральная частица, которая и уносит недостающую энергию. Он назвал эту частицу нейтрон (в последствии -- нейтрино). Он сформировал свое предложение в письме к Тюбингемскому научному конгрессу (под катом). Примечательны обращения «Dear radioactive ladies and gentlemen», «dear radioactives», а так же причина, по которой сам мистер Паули не явился на конгресс. У него ночью намечался бал. Дамы не будут ждать, пока ты тут новую частицу открываешь.}

\chapter{\textnormal{We \textsf{have} \texttt{several} \textsc{fonts}
  \textit{at} \textbf{disposal}}}
The serified roman font is used for the main body of the text.
\textit{Italics are typically used to denote emphasis or
quotations.} \texttt{The teletype font is typically used for source
code listings.} The \textbf{bold}, \textsc{small-caps} and
\textsf{sans-serif} variants of the base roman font can be used to
denote specific types of information.

\tiny We \scriptsize can \footnotesize also \small change \normalsize
the \large font \Large size, \LARGE although \huge it \Huge
is \huge usually \LARGE not \Large necessary.\normalsize

A wide variety of mathematical fonts is also available, such as: \[
  \mathrm{ABC}, \mathcal{ABC}, \mathbf{ABC}, \mathsf{ABC},
  \mathit{ABC}, \mathtt{ABC}
\] By loading the \textsf{amsfonts} packages, several additional
fonts will become available: \[
  \mathfrak{ABC}, \mathbb{ABC}
\] Many other mathematical fonts are available\footnote{
   Here: \url{http://tex.stackexchange.com/a/58124/70941}.
}.\\

\url{http://www.tug.dk/FontCatalogue/mathfonts.html}

\url{https://en.wikibooks.org/wiki/LaTeX/Mathematics}

%\url{ftp://tug.ctan.org/pub/tex-archive/info/Free_Math_Font_Survey/survey.html}

\url{http://www.tug.dk/FontCatalogue/}

\url{https://www.overleaf.com/learn/latex/Font_typefaces}

\end{document}
